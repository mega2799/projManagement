\chapter{Scoping}

% TODO: Completare la fase di Scoping

\section{Project Scoping Meeting}

% TODO: Elenco partecipanti e obiettivi del meeting

\section{Conditions of Satisfaction}

% TODO: Riferimento all'allegato e breve descrizione
% L'allegato in cui vengono riportati i criteri che devono essere soddisfatti
% per considerare il progetto completato con successo è reperibile al seguente link:
% \texttt{Allegato2.1-ConditionsOfSatisfaction.pdf}

\section{Project Overview Statement}

% TODO: Riferimento all'allegato e descrizione del POS
% Documento in cui è stato analizzato il progetto nella sua interezza dal punto
% di vista di problemi, opportunità, goal, obiettivi, criteri di successo,
% rischi, assunzioni e ostacoli.

\section{Risk Analysis}

% TODO: Descrizione della Risk Rating Matrix e approccio utilizzato

\section{Business Model Canvas}

% TODO: Descrizione delle 9 componenti critiche

\section{Analisi SWOT}

% TODO: Descrizione punti di forza, debolezze, opportunità e minacce

\section{Prototyping}

% TODO: Descrizione del processo di prototyping e iterazioni

\section{Requirements Breakdown Structure}

% TODO: Descrizione della struttura dei requisiti

\section{User Stories}

% TODO: Descrizione delle user stories e principio INVEST

\section{User Flow}

% TODO: Descrizione del flusso utente

\section{Project Management Life Cycle Models}

% TODO: Descrizione delle metodologie scelte per ogni sottosistema
% Giustificare le scelte in base alle caratteristiche di ogni sottosistema

\section{Approval Process}

% TODO: Descrizione del meeting di approvazione
