\chapter{Introduzione}

La digitalizzazione del patrimonio culturale e ludico tradizionale rappresenta una sfida affascinante nel panorama tecnologico contemporaneo. Il progetto \textbf{MaraffaOnline} nasce dall'incontro tra innovazione tecnologica e valorizzazione della tradizione, con l'obiettivo di portare un gioco di carte regionale italiano nel mondo digitale, preservandone l'essenza sociale e competitiva.

Questo documento descrive la gestione completa del progetto MaraffaOnline, dalla fase di scoping iniziale fino alla chiusura, seguendo le metodologie e gli strumenti del Project Management. Il progetto è stato realizzato da PlayHeritage Labs, uno spin-off universitario specializzato in cultural heritage gaming, su commissione della community "Maraffa Forever".

\section{L'Azienda: PlayHeritage Labs}

\textbf{PlayHeritage Labs} è uno spin-off universitario nato nel 2023 presso il Dipartimento di Informatica dell'Università di Bologna, sede di Cesena. L'azienda si colloca in un segmento di mercato innovativo: la digitalizzazione di giochi tradizionali e la valorizzazione del patrimonio ludico culturale attraverso tecnologie moderne.

Il team di PlayHeritage Labs è composto da cinque professionisti con competenze complementari:
\begin{itemize}
    \item \textbf{Gaming Development}: esperienza nello sviluppo di serious games e applicazioni ludiche
    \item \textbf{Cultural Heritage Digitale}: competenze nella preservazione e valorizzazione del patrimonio culturale immateriale
    \item \textbf{User Experience Design}: specializzazione nella progettazione di interfacce accessibili e intuitive
    \item \textbf{Project Management}: metodologie agili e tradizionali per la gestione di progetti software
    \item \textbf{Ricerca Accademica}: background scientifico con focus su gaming studies e interaction design
\end{itemize}

La mission aziendale è duplice: da un lato, creare prodotti digitali commercialmente sostenibili; dall'altro, contribuire alla ricerca accademica nel campo del cultural heritage gaming. MaraffaOnline rappresenta il primo progetto pilota di PlayHeritage Labs, con l'ambizione di validare il modello di business e l'approccio metodologico dell'azienda.

\section{Il Committente: Maraffa Forever}

Il committente del progetto è \textbf{"Maraffa Forever"}, una community informale di circa 150 appassionati del gioco di carte tradizionale Maraffa. Il gruppo è nato spontaneamente tra ex-studenti universitari della Romagna che, durante gli anni accademici, hanno condiviso la passione per questo gioco regionale.

Dopo la laurea, i membri della community si sono dispersi geograficamente in tutta Italia e all'estero per motivi professionali. Nonostante la distanza fisica, il gruppo ha mantenuto viva la comunità attraverso canali digitali (gruppo WhatsApp, forum online, social media), ma sentiva la mancanza dell'esperienza condivisa del gioco.

Riconoscendo l'assenza di soluzioni digitali di qualità per giocare a Maraffa online, la community ha deciso di auto-organizzarsi e raccogliere fondi attraverso un crowdfunding interno. Grazie al contributo volontario dei membri, è stato costituito un budget iniziale di \textbf{25.000 euro} da investire nello sviluppo di una piattaforma dedicata.

Il rappresentante ufficiale della community, designato come project sponsor, coordina le comunicazioni con PlayHeritage Labs e rappresenta gli interessi e le aspettative dei giocatori.

\section{Il Problema e l'Opportunità}

\subsection{Il Problema}

La Maraffa è un gioco di carte tradizionale diffuso principalmente in Romagna e in alcune zone dell'Italia centrale. Si gioca in quattro persone (due coppie), richiede strategia, memoria e comunicazione tra i partner. La dimensione sociale è fondamentale: il gioco è tradizionalmente praticato in contesti conviviali, nelle osterie, tra amici e in famiglia.

Il problema principale identificato dalla community "Maraffa Forever" è di natura \textbf{geografica e sociale}:
\begin{itemize}
    \item I membri del gruppo, un tempo compagni di università, sono ora sparsi in diverse città italiane ed europee
    \item Le occasioni di incontro fisico sono rare (1-2 volte all'anno)
    \item Le piattaforme online esistenti per giocare a Maraffa sono obsolete, poco funzionali o addirittura non più mantenute
    \item Le app disponibili offrono solo modalità single-player contro IA, perdendo completamente la dimensione sociale del gioco
    \item Non esiste una soluzione che consenta di giocare in tempo reale con amici specifici, ricreando l'esperienza del tavolo da gioco
\end{itemize}

\subsection{L'Opportunità}

Per PlayHeritage Labs, il progetto MaraffaOnline rappresenta un'opportunità strategica su più livelli:

\textbf{Validazione del Modello di Business}: essendo il primo progetto commerciale dello spin-off, MaraffaOnline permette di testare l'approccio alla digitalizzazione di giochi tradizionali, validare le metodologie di lavoro e creare un portfolio credibile.

\textbf{Ricerca Accademica}: il progetto offre materiale per pubblicazioni scientifiche su temi quali cultural heritage gaming, community-driven design, e user experience nella digitalizzazione di pratiche ludiche tradizionali. Il team leader del progetto intende utilizzare i risultati come caso di studio per la propria tesi di dottorato.

\textbf{Impatto Sociale}: contribuire alla preservazione di una tradizione ludica regionale e facilitare connessioni sociali tra persone geograficamente distanti rappresenta un valore aggiunto che va oltre il mero aspetto commerciale.

\textbf{Scalabilità}: il successo di MaraffaOnline potrebbe aprire la strada a progetti simili per altri giochi di carte regionali italiani (Briscola chiamata, Tresette, Scopone, ecc.), creando un ecosistema di giochi tradizionali digitalizzati.

\section{Obiettivo del Progetto}

L'obiettivo generale del progetto è la realizzazione di \textbf{MaraffaOnline}, una piattaforma web e mobile per giocare a Maraffa in multiplayer online, che ricrei fedelmente l'esperienza sociale del gioco dal vivo.

La piattaforma dovrà:
\begin{itemize}
    \item Permettere a quattro giocatori di giocare partite in tempo reale, indipendentemente dalla loro posizione geografica
    \item Implementare fedelmente le regole della Maraffa tradizionale romagnola
    \item Offrire un'interfaccia moderna, intuitiva e accessibile, adatta al target di giovani adulti (25-45 anni)
    \item Integrare funzionalità sociali: creazione di stanze private, chat, sistema di amicizie, statistiche personali
    \item Garantire un'esperienza fluida sia da desktop che da dispositivi mobili (responsive design o app nativa)
    \item Assicurare scalabilità, sicurezza e affidabilità tecnica
\end{itemize}

Il progetto ha una durata prevista di \textbf{6 mesi} e un budget di \textbf{25.000 euro}, forniti dal committente. Il team di progetto è composto dai cinque membri di PlayHeritage Labs, con ruoli e responsabilità distribuiti secondo competenze specifiche.

Il successo del progetto sarà misurato non solo in termini di completamento tecnico delle funzionalità, ma anche attraverso il livello di soddisfazione della community committente e l'adozione effettiva della piattaforma da parte dei giocatori target.
